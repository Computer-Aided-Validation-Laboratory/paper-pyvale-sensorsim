\documentclass[11pt, a4paper, oneside, onecolumn]{article}
\usepackage[left=2cm, right=2cm, top=2cm, bottom=2cm]{geometry}
\usepackage[utf8]{inputenc}
\usepackage{cite}
\usepackage{amsmath,amsfonts,amsthm,bm}
\usepackage{array, booktabs, makecell}
\usepackage{graphicx}
\usepackage{epstopdf} 
\usepackage{gensymb}
\usepackage{caption}
\usepackage{subcaption}
\usepackage{tablefootnote}
\usepackage{verbatim}
\usepackage{hyperref}
\usepackage{url}
\usepackage[capitalise]{cleveref}
	\crefname{equation}{equation}{equations}
	\crefname{figure}{figure}{figures}	
	\crefname{table}{table}{tables}
\usepackage[flushleft]{threeparttable}
\usepackage [english]{babel}
\usepackage [autostyle, english = american]{csquotes}
\MakeOuterQuote{"}
\newcommand{\ra}[1]{\renewcommand{\arraystretch}{#1}}
\newcommand*{\rttensor}[1]{\underline{\boldsymbol{#1}}}
\newcommand*{\rttensortwo}[1]{\underline{\underline{\boldsymbol{#1}}}}

%----------------------------------------------------------------------------------
\title{pyvale: }

\author{Lloyd Fletcher\textsuperscript{1,2} \and Fabrice Pierron\textsuperscript{1,3}\\
\textsuperscript{1}Faculty of Engineering and Physical Sciences, University of Southampton, UK \\
\textsuperscript{2}UKAEA, Rotherham, UK \\
\textsuperscript{3}MatchID NV, Ghent, Belgium \\
}

\date{}

\begin{document}

\maketitle

\begin{abstract}
TODO
% 100 words

Keywords: 
\end{abstract}

\section*{Metadata}
\label{}
\textit{The ancillary data table~\ref{codeMetadata} is required for the sub-version of the codebase. Please replace the italicized text in the right column with the correct information about your current code and leave the left column untouched.}

\begin{table}[!h]
\begin{tabular}{|l|p{6.5cm}|p{6.5cm}|}
\hline
\textbf{Nr.} & \textbf{Code metadata description} & \textbf{Metadata} \\
\hline
C1 & Current code version & For example v42 \\
\hline
C2 & Permanent link to code/repository used for this code version & For example: \url{https://github.com/mozart/mozart2} \\
\hline
C3  & Permanent link to Reproducible Capsule & For example: \url{https://codeocean.com/capsule/0270963/tree/v1}\\
\hline
C4 & Legal Code License   & All software and code must be released under one of the pre-approved licenses listed in the \href{https://www.elsevier.com/journals/softwarex/2352-7110/guide-for-authors}{Guide for Authors}, such as Apache License, GNU General Public License (GPL) or MIT License. Write the name of the license you’ve chosen here. \\
\hline
C5 & Code versioning system used & For example: svn, git, mercurial,
                                   etc. (put none if none is used) \\
\hline
C6 & Software code languages, tools, and services used & For example: C++, python, r, MPI, OpenCL, etc. \\
\hline
C7 & Compilation requirements, operating environments \& dependencies & \\
\hline
C8 & If available Link to developer documentation/manual & For example: \url{http://mozart.github.io/documentation/} \\
\hline
C9 & Support email for questions & \\
\hline
\end{tabular}
\caption{Code metadata (mandatory)}
\label{codeMetadata} 
\end{table}

\textit{Optionally, you can provide information about the current executable
software version filling in the left column of
Table~\ref{executabelMetadata}. Please leave the first column as it is.}

\begin{table}[!h]
\begin{tabular}{|l|p{6.5cm}|p{6.5cm}|}
\hline
\textbf{Nr.} & \textbf{(Executable) software metadata description} & \textbf{Please fill in this column} \\
\hline
S1 & Current software version & For example 1.1, 2.4 etc. \\
\hline
S2 & Permanent link to executables of this version  & For example: \url{https://github.com/combogenomics/DuctApe/releases/tag/DuctApe-0.16.4} \\
\hline
S3  & Permanent link to Reproducible Capsule & \\
\hline
S4 & Legal Software License & List one of the approved licenses \\
\hline
S5 & Computing platforms/Operating Systems & For example Android, BSD, iOS, Linux, OS X, Microsoft Windows, Unix-like , IBM z/OS, distributed/web based etc. \\
\hline
S6 & Installation requirements \& dependencies & \\
\hline
S7 & If available, link to user manual - if formally published include a reference to the publication in the reference list & For example: \url{http://mozart.github.io/documentation/} \\
\hline
S8 & Support email for questions & \\
\hline
\end{tabular}
\caption{Software metadata (optional)}
\label{executabelMetadata} 
\end{table}

\section{Motivation and significance}
TODO \cite{gaston_moose_2009}

\textit{In this section, we want you to introduce the scientific background and the motivation for developing the software.}

\begin{itemize}
    \item \textit{Explain why the software is important and describe the exact (scientific) problem(s) it solves.}
    \item \textit{Indicate in what way the software has contributed (or will contribute in the future) to the process of scientific discovery; if available, please cite a research paper using the software.}
    \item \textit{Provide a description of the experimental setting. (How does the user use the software?)}
    \item \textit{Introduce related work in literature (cite or list algorithms used, other software etc.).}
\end{itemize}

\section{Software description}

\textit{Describe the software. Provide enough detail to help the reader understand its impact. }

\subsection{Software architecture}
\textit{  Give a short overview of the overall software architecture; provide a pictorial overview where possible; for example, an image showing the components. If necessary, provide implementation details.}

 \subsection{Software functionalities}
\textit{  Present the major functionalities of the software.}
  
 \subsection{Sample code snippets analysis (optional)}


\section{Illustrative examples}

\textit{Provide at least one illustrative example to demonstrate the major
functions of your software/code.}

\textit{\textbf{Optional}: you may include one explanatory  video or screencast that will appear next to your article, in the right hand side panel. Please upload any video as a single supplementary file with your article. Only one MP4 formatted, with 150MB maximum size, video is possible per article. Recommended video dimensions are 640 x 480 at a maximum of 30 frames / second. Prior to submission please test and validate your .mp4 file at  \url{http://elsevier-apps.sciverse.com/GadgetVideoPodcastPlayerWeb/verification} . This tool will display your video exactly in the same way as it will appear on ScienceDirect. }

\section{Impact}
\textit{This is the main section of the article and reviewers will weight it appropriately.
Please indicate:}
\begin{itemize}
    \item \textit{Any new research questions that can be pursued as a result of your software.}
    \item \textit{In what way, and to what extent, your software improves the pursuit of existing research questions.}
    \item \textit{Any ways in which your software has changed the daily practice of its users.}
    \item \textit{How widespread the use of the software is within and outside the intended user group (downloads, number of users if your software is a service, citable publications, etc.).}
    \item \textit{How the software is being used in commercial settings and/or how it has led to the creation of spin-off companies.}
    \end{itemize}
\textit{Please note that points 1 and 2 are best demonstrated by
  references to citable publications.}

\section{Conclusions}


%----------------------------------------------------------------------------------
\section*{Acknowledgements} \label{sec:acknowledgements}
TODO

%----------------------------------------------------------------------------------
\section*{Data Provision} \label{sec:dataProvision}
\noindent All data supporting this study are openly available from XXXX repository at: http://dx.doi.org/XXXXX. The digital dataset contains the following:

\begin{enumerate}
\item TODO
\end{enumerate}


%----------------------------------------------------------------------------------
\bibliographystyle{unsrt} 
\bibliography{pyvale-paper-sensorsim}   % name your BibTeX data base

\end{document}

