\documentclass[11pt, a4paper, oneside, onecolumn]{article}
\usepackage[left=2cm, right=2cm, top=2cm, bottom=2cm]{geometry}
\usepackage[utf8]{inputenc}
\usepackage{cite}
\usepackage{amsmath,amsfonts,amsthm,bm}
\usepackage{array, booktabs, makecell}
\usepackage{graphicx}
\usepackage{epstopdf} 
\usepackage{gensymb}
\usepackage{caption}
\usepackage{subcaption}
\usepackage{tablefootnote}
\usepackage{verbatim}
\usepackage{hyperref}
\usepackage{url}
\usepackage[capitalise]{cleveref}
	\crefname{equation}{equation}{equations}
	\crefname{figure}{figure}{figures}	
	\crefname{table}{table}{tables}
\usepackage[flushleft]{threeparttable}
\usepackage [english]{babel}
\usepackage [autostyle, english = american]{csquotes}
\MakeOuterQuote{"}
\newcommand{\ra}[1]{\renewcommand{\arraystretch}{#1}}
\newcommand*{\rttensor}[1]{\underline{\boldsymbol{#1}}}
\newcommand*{\rttensortwo}[1]{\underline{\underline{\boldsymbol{#1}}}}

%----------------------------------------------------------------------------------
\title{pyvale: A virtual engineering laboratory for simulating sensor measurement uncertainties}

\author{Lloyd Fletcher\textsuperscript{1} \and Joel Hirst\textsuperscript{1}\\
\textsuperscript{1}UK Atomic Energy Authority, Fusion Technology Facilities, Rotherham, UK\\
\textsuperscript{2}UK Atomic Energy Authority, Culham Campus, Culham, UK\\
}

\date{}

\begin{document}

\maketitle

\begin{abstract}
`pyvale` is a simulation package... virtual engineering laboratory  
% 100 words

%TODO
% - Why?
% - What?
% - How?
% - Impact?

Keywords: 
\end{abstract}

\section*{Metadata}
\label{}
\begin{table}[!h]
\begin{tabular}{|l|p{6.5cm}|p{6.5cm}|}
\hline
\textbf{Nr.} & \textbf{Code metadata description} & \textbf{Metadata} \\
\hline
C1 & Current code version & 2025.X.X \\
\hline
C2 & Permanent link to code/repository used for this code version & \url{https://github.com/Computer-Aided-Validation-Laboratory/pyvale} \\
\hline
C3  & Permanent link to Reproducible Capsule & For example: TODO\\
\hline
C4 & Legal Code License   & MIT License \\
\hline
C5 & Code versioning system used & git \\
\hline
C6 & Software code languages, tools, and services used & python \\
\hline
C7 & Compilation requirements, operating environments \& dependencies & Cross platform distributed on the python package index, pypi: `pip install pyvale` \\
\hline
C8 & If available Link to developer documentation/manual & \url{https://computer-aided-validation-laboratory.github.io/pyvale/index.html} \\
\hline
C9 & Support email for questions & lloyd.fletcher@ukaea.uk \\
\hline
\end{tabular}
\caption{Code metadata (mandatory)}
\label{codeMetadata} 
\end{table}

\section{Motivation and significance}

Qualification of fusion technology is reliant on simulations to predict the performance of components in extreme (e.g., thermal and electromagnetic) and untestable (e.g., fusion neutron fluxes) environments. Enabling the use of simulations for risk-informed decision making requires that they are validated over testable domains to reduce uncertainty when extrapolating into irradiated conditions. The cost of performing large-scale validation tests on a complex components such as a breeder blankets will be on the order of £M's. Therefore, significant cost and risk reduction can be achieved by maximising the information obtained from an optimised set of targeted experiments.

A key parameter of validation experiments is the deployment of sensor arrays to measure the components response. There are currently no software tools available that can simulate and optimise the placement of diverse arrays of sensors for multi-physics conditions with realistic constraints (e.g., cost, reliability, and accuracy). Such a tool would have immediate benefits for reducing costs of the experimental programme required to qualify fusion components such as the breeder blankets and divertors [REFs].

Simulation validation is a fundamental problem of scientific and engineering computing [REFS] and it is particularly challenging for the multi-physics environments components are subjected to in fusion reactors [REFs]. Experimental validation  

We are developing this software in parallel to a research programme focusing on .

We envisage multiple applications of `pyvale`: testing validation metrics with known ground truth while modelling systematic and random measurement errors.  

Here we describe the core sensor simulation engine of `pyvale`

  

TODO \cite{gaston_moose_2009}

%TODO
% - pyvale is important as it allows engineers to test and model sensor deployments analysing random and systematic errors.
% - pyvale is physics agnostic: a sensor samples a physical field which can be a scalar, vector or tensor field. Multiple sensor arrays
% - pyvale performs Monte-Carlo Simulation in line with the GUM
% - pyvale takes inspiration from the digital image correlation community where synthetic image deformation simulations have been key to understanding and mitigating
% - pyvale is cross platform and targeted for ease of use by experimental and simulation engineers (no need to build from source)   

\textit{In this section, we want you to introduce the scientific background and the motivation for developing the software.}

\begin{itemize}
    \item \textit{Explain why the software is important and describe the exact (scientific) problem(s) it solves.}
    \item \textit{Indicate in what way the software has contributed (or will contribute in the future) to the process of scientific discovery; if available, please cite a research paper using the software.}
    \item \textit{Provide a description of the experimental setting. (How does the user use the software?)}
    \item \textit{Introduce related work in literature (cite or list algorithms used, other software etc.).}
\end{itemize}

\section{Software description}

\textit{Describe the software. Provide enough detail to help the reader understand its impact. }

\subsection{Software architecture}
\textit{  Give a short overview of the overall software architecture; provide a pictorial overview where possible; for example, an image showing the components. If necessary, provide implementation details.}

 \subsection{Software functionalities}
\textit{  Present the major functionalities of the software.}
  
 \subsection{Sample code snippets analysis (optional)}


\section{Illustrative examples}

\textit{Provide at least one illustrative example to demonstrate the major
functions of your software/code.}

\section{Impact}
\textit{This is the main section of the article and reviewers will weight it appropriately.
Please indicate:}
\begin{itemize}
    \item \textit{Any new research questions that can be pursued as a result of your software.}
    \item \textit{In what way, and to what extent, your software improves the pursuit of existing research questions.}
    \item \textit{Any ways in which your software has changed the daily practice of its users.}
    \item \textit{How widespread the use of the software is within and outside the intended user group (downloads, number of users if your software is a service, citable publications, etc.).}
    \item \textit{How the software is being used in commercial settings and/or how it has led to the creation of spin-off companies.}
    \end{itemize}
\textit{Please note that points 1 and 2 are best demonstrated by
  references to citable publications.}
  
Sensor simulation has the potential to impact a wide range of engineering fields. `pyvale` is a new 

We have specifically designed pyvale 

Future modules for pyvale will include 

\section{Conclusions}


%----------------------------------------------------------------------------------
\section*{Acknowledgements} \label{sec:acknowledgements}
Dr Lloyd Fletcher acknowledges support from UKRI through the Future Leaders Fellowship scheme (grant ).

\section*{CRediT Authorship Statement}
TODO


%----------------------------------------------------------------------------------
\bibliographystyle{unsrt} 
\bibliography{pyvale-paper-sensorsim}   % name your BibTeX data base

\end{document}

